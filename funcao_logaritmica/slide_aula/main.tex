\documentclass[12pt]{beamer}
\usepackage[table,xcdraw]{xcolor}
\usetheme{Madrid}
\usecolortheme{seahorse}

\usepackage{ragged2e}
\usepackage[utf8]{inputenc}
\usepackage[brazil]{babel}
\usepackage[T1]{fontenc}
\usepackage{amsmath}
\usepackage{amsfonts}
\usepackage{amssymb}
\usepackage{graphicx}
\usepackage{setspace}
\setbeamertemplate{caption}[numbered]
\usepackage{hyperref}
\setbeamercovered{transparent} 
\setbeamertemplate{navigation symbols}{}
\usepackage[table,xcdraw]{xcolor}
\usepackage{ragged2e}
\newcommand{\euler}{e}

\newcounter{exercicio}
\setcounter{exercicio}{0}

\newcommand{\exercicio}{
	\stepcounter{exercicio}
	\textbf{Exercício \theexercicio.}
}

%referência e citação
\usepackage[bibstyle=abnt-numeric,citestyle=authoryear,backend=biber,]{biblatex}
\addbibresource{ref.bib}

\addtobeamertemplate{block alerted}{}{\justifying}

\author[Joao Victor \& Prof. Luiz Claudio]{% 
Joao Victor\inst{1} \and Prof. Luiz Claudio\inst{2}}
\title[Função Logarítmica]{Função Logarítmica}
\institute[]{% 
  \textsuperscript{1} Instituto Federal do Amazonas \\
  \textsuperscript{2} EETI Bilíngue Gilberto Mestrinho de Medeiros Raposo
}

\titlegraphic{\includegraphics[width=0.2\textwidth]{imagens/logo_3.jpg}}
\date{\today} 


%\subject{}

% ---------------------------------------------------------

\begin{document}
\onehalfspacing 
\justifying 

\begin{frame}
    \titlepage
\end{frame}

\begin{frame}{Sumário}
    \tableofcontents
\end{frame}


\section{Introdução}

	\begin{frame}{Introdução (logaritmo)}
		
			\justifying
			Suponhamos que um caminhão custe hoje R$\$ 100.000,00$ e sofra uma desvalorização de $10\%$ por ano de uso. Depois de quanto tempo de uso o valor do veículo será igual a R$\$ 20.000,00$?
		
	\end{frame}
	
	
\section{Definição}

	\begin{frame}{Definição}
		Sendo $a$ e $b$ números reais e positivos, com $a \neq 1$, chama-se \textbf{logaritmo de b na base a} o expoente $x$ ao qual se deve elevar a base $a$ de modo que a potência $a^{x}$ seja igual a $b$.
		
		$$\log_{a}b=x \Leftrightarrow a^{x}=b$$
		
		\uncover<2->{
		\begin{block}{Dizemos que:}
			\begin{itemize}
				\item $a$ é a \textbf{base} do logaritmo;
				\item $b$ é o \textbf{logaritmando};
				\item $x$ é o \textbf{logaritmo}.
			\end{itemize}
		\end{block}
		}
	\end{frame}

	\begin{frame}{Definição}
		
		\begin{figure}
			\centering
			\includegraphics[scale=0.5]{imagens/ex-poliedro.png}
			\caption{Sistema de Ensino Poliedro}
		\end{figure}
	
	\end{frame}

    \begin{frame}{Exemplos}
       
       Vejamos alguns exemplos de logaritmos:
       
       \begin{itemize}
       		\item [a)] $\log_{2}(8)=3, \ \text{pois} \ 2^{3}=8$ \\ 
       		\uncover<2->{\item [b)] $\log_{3}9=2, \ \text{pois} \ 3^{2}=9 $} \\
       		\uncover<3->{\item [c)] $\log_{4}1=0, \ \text{pois} \ 4^{0}=1 $} \\
       		\uncover<4->{\item [d)] $\log_{\frac{1}{2}}8=-3, \ \text{pois} \ \left( \dfrac{1}{2} \right)^{-3}=8 $} \\
       		\uncover<5->{\item [e)] $\log_{0,5}0,25=2, \ \text{pois} \ \left(0,5 \right)^{2}=0,25 $}
       	\end{itemize}
    \end{frame}

    \begin{frame}{Aplicação}
    	
    	\begin{alertblock}{\exercicio Calcule, por meio da definição:}
    		 
    		\begin{itemize}
    			\item [a)] $\log_{\sqrt[3]9}3$ \\
    			\item [b)] $\log_{16}0,25$
    		\end{itemize}
    		
    	\end{alertblock}
    		
    	\begin{alertblock}{\exercicio}
    		Qual o número real $x$ em que $\log_{x}4=-2$?
    	\end{alertblock}
    	
       
    \end{frame}
    
    
    \begin{frame}{Convenção importante}
    	
    	\begin{block}{Logaritmo decimal}
    		\justifying
    		Isto é, quando a base do logaritmo for 10, podemos omiti-la da notação:
    		\begin{itemize}
    			\item $\log1 = \log_{10}1=0$
    			\item $\log1000=\log_{10}1000=3$
    		\end{itemize}
    	\end{block}
    	
    	\uncover<2->{
    	\begin{block}{Logaritmo natural (Logaritmo natural)}
    		\justifying
    		Já quando a base do logaritmo é o número de Euler, $\euler \approx 2,7183$, a notação utilizada é $\ln$:
    		\begin{itemize}
    			\item $\ln e = \log_{\euler}\euler = 1$
    			\item $\ln \euler^{3}=\log_{\euler}\euler^{3}=3$
    		\end{itemize}
    	\end{block}
    	}
    	
    \end{frame}

\section{Consequências}
	
	\begin{frame}{Consequências}
		Sejam $a,b$ e $c$ números reais com $0<a\neq 1, b>0 \ \text{e} \ c>0$. Decorrem da definição de logaritmo as seguintes propriedades (P):
  
	
		\begin{block}{P1: O logaritmo de 1 em qualquer base $a$ é igual a $0$.}
			$$\log_{a}1=0,\  \text{pois} \ a^{0}=1$$
		\end{block}
		
		\uncover<2->{\begin{block}{P2: O logaritmo da base, qualquer que seja ela, é igual a 1.}
			$$\log_{a}a=1,\  \text{pois} \ a^{1}=a$$
		\end{block}}
		
		\uncover<3->{\begin{block}{P3: A potência de base $a$ e expoente $\log_{a}b$ é igual a b.}
			$$a^{\log_{a}b}=b$$
		\end{block}}
	
	\end{frame}
	
	\begin{frame}{Consequências}
		Sejam $a,b$ e $c$ números reais com $0<a\neq 1, b>0 \ \text{e} \ c>0$. Decorrem da definição de logaritmo as seguintes propriedades (P):
		
		
		\begin{block}{P4:}
			\begin{itemize}
				\justifying
				\item Se dois logaritmos em uma mesma base são iguais, então os logaritmandos também são iguais. $$ \log_{a}b=\log_{a}c \Rightarrow  b=c$$
			
				\item Reciprocamente, se dois números reais positivos são iguais, seus logaritmos em uma mesma base também são iguais.
				
			\end{itemize}
		\end{block}
	\end{frame}
 
	\begin{frame}{Aplicação}
		\begin{alertblock}{\exercicio}
			Qual é o valor de $9^{\log_{3}5}$?
		\end{alertblock}
		
		\begin{alertblock}{\exercicio}
			Vamos calcular o número real x tal que $\log_{5}(2x+1)=\log_{5}(x+3)$
		\end{alertblock}
		
		
		\begin{alertblock}{\exercicio}
			Qual é o valor de cada uma das expressões seguintes?
			
			\begin{itemize}
				\item [a)] $\log_{5}5+\log_{3}1-\log 10$
				\item [b)] $\log_{\frac{1}{4}}4+\log_{4}\frac{1}{4}$
				\item [c)] $3^{\log_{3}2}+2^{\log_{2}3}$
			\end{itemize}
		\end{alertblock}
		
	\end{frame}
	
\section{Propriedades operatórias}
	
	\begin{frame}{Propriedades operatórias (PO)}
		Vamos  agora estudar três propriedades operatórias envolvendo logaritmos. \\
		
		\begin{block}{PO1: Logaritmo do produto}
			\justifying
			Em qualquer base, o logaritmo do produto de dois números reais e positivos é igual à soma dos logaritmos de cada um deles, isto é, se $0<a\neq 1, \ b>0 \ \text{e} \ c>0$, então: $$\log_{a}(b.c)=\log_{a}b+\log_{a}c$$
		\end{block} 
		
		Por exemplo: $$\log_{2}6=\log_{2}(2.3)=\log_{2}2+\log_{2}3=1+\log_{2}3$$
	\end{frame}
	
	\begin{frame}{Propriedades operatórias (PO)}
		
		
		\begin{block}{PO2: Logaritmo do quociente}
			\justifying
			Em qualquer base, o logaritmo do quociente de dois números reais e positivos é igual à diferença entre o logaritmo do numerador e o logaritmo do denominador, isto é, se $0<a\neq 1, \ b>0 \ \text{e} \ c>0$, então: $$\log_{a}\left( \frac{b}{c} \right)=\log_{a}b-\log_{a}c$$
		\end{block} 
		
		Por exemplo: $$\log\left(\frac{3}{100} \right)=\log3-\log100=\log3-2$$
	\end{frame}
	
	\begin{frame}{Propriedades operatórias (PO)}
		
		
		\begin{block}{PO3: Logaritmo da potência}
			\justifying
			Em qualquer base, o logaritmo de uma potência de base real e positiva é igual ao produto do expoente pelo logaritmo da base da potência, isto é, se $0<a\neq 1, \ b>0 \ \text{e} \ r \in \mathbb{R}$, então: $$\log_{a}b^{r}=r\log_{a}b$$
		\end{block} 
		
		Por exemplo: $$\log_{5}27=\log_{5}3^{3}=3\log_{5}3$$
	\end{frame}
	
	
	\begin{frame}{Aplicação}
		\begin{alertblock}{\exercicio}
			\justifying
			Calcular o valor de $\log_{b}\left(x^{2}.y\right)$, sabendo que $\log_{b}x=3$ e $\log_{b}y=-4 \left(x>0, y>0 \ \text{e} \ 0<b\neq 1 \right)$
		\end{alertblock}
		
		\begin{alertblock}{\exercicio}
			\justifying
			Qual é a expressão E cujo desenvolvimento logarítmico (em base 10) é $\log\text{E} = 1+\log\text{a}+2\log\text{b}-\log\text{c}$, com a, b e c números reais e positivos?
		\end{alertblock}
	
	
		\begin{alertblock}{\exercicio}
			\justifying
			Admitindo que $\log2=\text{a}$ e $\log3=\text{b}$, obter o valor de $\log0,48$, em função de $a$ e $b$.
		\end{alertblock}
	
	\end{frame}

%exampleblock


\end{document}