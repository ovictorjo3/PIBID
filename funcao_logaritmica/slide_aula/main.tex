\documentclass[12pt]{beamer}
\usepackage[table,xcdraw]{xcolor}
\usetheme{Madrid}
\usecolortheme{seahorse}

\usepackage{ragged2e}
\usepackage[utf8]{inputenc}
\usepackage[brazil]{babel}
\usepackage[T1]{fontenc}
\usepackage{amsmath}
\usepackage{amsfonts}
\usepackage{amssymb}
\usepackage{graphicx}
\usepackage{setspace}
\setbeamertemplate{caption}[numbered]
\usepackage{hyperref}
\setbeamercovered{transparent} 
\setbeamertemplate{navigation symbols}{}
\usepackage[table,xcdraw]{xcolor}
\usepackage{ragged2e}

\newcounter{exercicio}
\setcounter{exercicio}{0}

\newcommand{\exercicio}{
	\stepcounter{exercicio}
	\textbf{Exercício \theexercicio.}
}

%referência e citação
\usepackage[bibstyle=abnt-numeric,citestyle=authoryear,backend=biber,]{biblatex}
\addbibresource{ref.bib}

\addtobeamertemplate{block alerted}{}{\justifying}

\author[Joao Victor \& Prof. Luiz Claudio]{% 
Joao Victor\inst{1} \and Prof. Luiz Claudio\inst{2}}
\title[Função Logarítmica]{Função Logarítmica}
\institute[]{% 
  \textsuperscript{1} Instituto Federal do Amazonas \\
  \textsuperscript{2} EETI Bilíngue Gilberto Mestrinho de Medeiros Raposo
}

\titlegraphic{\includegraphics[width=0.2\textwidth]{imagens/logo_3.jpg}}
\date{\today} 


%\subject{}

% ---------------------------------------------------------

\begin{document}
\onehalfspacing 
\justifying 

\begin{frame}
    \titlepage
\end{frame}

\begin{frame}{Sumário}
    \tableofcontents
\end{frame}


\section{Introdução}

	\begin{frame}{Introdução (logaritmo)}
		
			\justifying
			Suponhamos que um caminhão custe hoje R$\$ 100.000,00$ e sofra uma desvalorização de $10\%$ por ano de uso. Depois de quanto tempo de uso o valor do veículo será igual a R$\$ 20.000,00$?
		
	\end{frame}
	
	
\section{Definição}

	\begin{frame}{Definição}
		Sendo $a$ e $b$ números reais e positivos, com $a \neq 1$, chama-se \textbf{logaritmo de b na base a} o expoente $x$ ao qual se deve elevar a base $a$ de modo que a potência $a^{x}$ seja igual a $b$.
		
		$$\log_{a}b=x \Leftrightarrow a^{x}=b$$
		
		\uncover<2->{
		\begin{block}{Dizemos que:}
			\begin{itemize}
				\item $a$ é a \textbf{base} do logaritmo;
				\item $b$ é o \textbf{logaritmando};
				\item $x$ é o \textbf{logaritmo}.
			\end{itemize}
		\end{block}
		}
	\end{frame}

    \begin{frame}{Exemplos}
       
       Vejamos alguns exemplos de logaritmos:
       
       \begin{itemize}
       		\item [a)] $\log_{2}(8)=3, \ \text{pois} \ 2^{3}=8$ \\ 
       		\uncover<2->{\item [b)] $\log_{3}9=2, \ \text{pois} \ 3^{2}=9 $} \\
       		\uncover<3->{\item [c)] $\log_{4}1=0, \ \text{pois} \ 4^{0}=1 $} \\
       		\uncover<4->{\item [d)] $\log_{\frac{1}{2}}8=-3, \ \text{pois} \ \left( \dfrac{1}{2} \right)^{-3}=8 $} \\
       		\uncover<5->{\item [e)] $\log_{0,5}0,25=2, \ \text{pois} \ \left(0,5 \right)^{2}=0,25 $}
       	\end{itemize}
    \end{frame}

    \begin{frame}{Aplicação}
    	
    	\begin{alertblock}{\exercicio Calcule, por meio da definição:}
    		 
    		\begin{itemize}
    			\item [a)] $\log_{\sqrt[3]9}3$ \\
    			\item [b)] $\log_{16}0,25$
    		\end{itemize}
    		
    	\end{alertblock}
    		
    		
    	\begin{alertblock}{\exercicio}
    		Qual o número real $x$ em que $\log_{x}4=-2$?
    	\end{alertblock}
    	
       
    \end{frame}
    
    
    \begin{frame}{Convenção importante}
    	Convencionou-se que, ao escrevermos o logaritmo de um número com a base omitida, estamos nos referindo ao logaritmo desse número em base 10, isto é:
    	
    	$$\log x = \log_{10}x$$
    	
    	\begin{block}{Assim, por exemplo:}
    		\begin{itemize}
    			\item $\log 10.000 = 4,\  \text{pois} \ 10^{4}=10.000 $ \\
    			\item $\log\dfrac{1}{1000}  = -3, \ \text{pois} \ 10^{-3}=\dfrac{1}{1000}$ \\
    		\end{itemize}
    	\end{block}
    	
    	\uncover<2->{Os logaritmos em base 10 são conhecidos como \textbf{logaritmos decimais}.}
    	
    \end{frame}

    \begin{frame}{Tipos de triângulos}
        Quanto aos lados, os triângulos se classificam em:

        \begin{center}
            \includegraphics[scale=0.5]{imagens/triangulos.png}
        \end{center}
    \end{frame}
\section{Desigualdade Triangular}

    \begin{frame}{Desigualdade triangular}
        \begin{alertblock}{Definição}
        \justifying
            Em todo triângulo, a soma dos comprimentos de dois lados é maior que o comprimento do terceiro lado. 
        \end{alertblock}
        
        \pause
        
        \begin{block}{Aplicação}
            Existe triângulo cujos lados medem 5, 8 e 16? Por quê?
        \end{block}

    \end{frame}
    
\section{Verificação: é triângulo?}

    \begin{frame}{Probabilidade Geométrica - Problema do macarrão}
    \begin{minipage}{\textwidth}
        \centering
        \fbox{\begin{minipage}{0.9\textwidth}
            \vspace{0.3cm}
            \footnotesize
            
            \textbf{O grande desafio:} \\
            1. Pegue UM fio de espaguete \\
            2. QUEBRE-EM em 3 partes \textbf{arbitrariamente} \\
            3. Meça os comprimentos \textbf{a, b, c} \\
            4. Teste as desigualdades (marque \checkmark\ ou $\times$) \\
            5. Agora tente FORMAR o triângulo \\
            6. Se \textbf{TODAS} as desigualdades forem verdadeiras $\rightarrow$ triângulo possível! \\
            7. Se \textbf{UMA} for falsa $\rightarrow$ triângulo impossível! \\
            
            \textbf{Consegue prever antes de tentar montar?} \\
            \textbf{Regra da Desigualdade Triangular:} \\
            $a + b > c$ \textbf{e} $a + c > b$ \textbf{e} $b + c > a$
            \vspace{0.3cm}
        \end{minipage}}
    \end{minipage}
\end{frame}

    \begin{frame}{Aplicação em sala}
        \begin{table}[h]
        \centering
        \small
        \setlength{\tabcolsep}{8pt}
        \renewcommand{\arraystretch}{1.5}

            \begin{tabular}{|c|c|c||c|c|c||c|}
                \hline
                \textbf{a} & \textbf{b} & \textbf{c} & \textbf{a + b > c} & \textbf{a + c > b} & \textbf{b + c > a} & \textbf{Triângulo?} \\
                \hline
                & & & $\square$ & $\square$ & $\square$ & \\
                \hline
                & & & $\square$ & $\square$ & $\square$ & \\
                \hline
                & & & $\square$ & $\square$ & $\square$ & \\
                \hline
                & & & $\square$ & $\square$ & $\square$ & \\
                \hline
                & & & $\square$ & $\square$ & $\square$ & \\
                \hline
                & & & $\square$ & $\square$ & $\square$ & \\
                \hline
            \end{tabular}
        \end{table}
    \end{frame}

\section{Problema, probleminhas e \textit{problemão} (OBMEP)}

\begin{frame}{Probleminha: Quem andou mais?}

    \begin{exampleblock}{\textbf{Desafio 01.}}
        Ruas retas e compridas ligam as casas dos amigos Bruno, Francimar e Robério.

    \begin{itemize}\justifying
        \item Francimar, em sua caminhada matinal, saiu de sua casa e andou até a casa de Bruno. Em seguida, prosseguiu para a casa de Robério e depois voltou para sua casa.
        \item Mais tarde, Robério, muito concentrado com um problema de matemática, foi até a casa de Bruno e voltou para sua casa.
    \end{itemize} Sem conhecer as distâncias entre as casas, é possível saber quem andou mais?
    
    \end{exampleblock}

\end{frame}

\begin{frame}{Probleminha: Brincando com lápis}

    \begin{exampleblock}{\textbf{Desafio 02.}}
        \justifying
        Ana Paula tinha 2 lápis em mãos, cujos comprimentos eram de 5,8 cm e 11,4 cm, respectivamente. Com esses 2 lápis e um terceiro, entre os que tinha em seu estojo, ela começou a formar triângulos que tivessem os seus lápis como lados. Logo ela percebeu que com alguns dos lápis do estojo não era possível formar um triângulo. 

    \vspace{2mm} 

    Determine para que comprimentos do terceiro lápis Ana Paula conseguirá formar um triângulo.
    \end{exampleblock}
   
\end{frame}

\begin{frame}{Problema de Gincana: Isso não é perímetro}
\begin{exampleblock}{\textbf{Desafio 03.}}
    Se $\overline{AB}+\overline{BC}=18$, então o perímetro do triângulo $ABC$ \textbf{NÃO} pode ser:

    \begin{itemize}
        \item [a)] 33
        \item [b)] 34
        \item [c)] 35
        \item [d)] 36
        \item [e)] Nenhuma das respostas anteriores
    \end{itemize}
\end{exampleblock}
    
\end{frame}

\begin{frame}{Unicamp - 2024}

    \begin{exampleblock}{\textbf{Desafio 04.}}
        \justifying
        Joaquim estava brincando com um graveto, quando acertou uma parede e o graveto se partiu em três pedaços, de comprimentos a, b, c, com $a \leq b \leq c$. Ele recolheu os pedaços e tentou construir um triângulo cujos lados seriam exatamente os pedaços do graveto: \textbf{não foi possível}. Sabendo que o graveto tinha $50\ cm$ de comprimento e que $b = a + 2$, qual é o maior valor possível de a?

        \begin{itemize}
        \item [a)] 9,5 cm
        \item [b)] 10,5 cm
        \item [c)] 11,5 cm %
        \item [d)] 12,5 cm
    \end{itemize}
    \end{exampleblock}
   
\end{frame}

\begin{frame}{Problemão: Probabilidade com macarrão}

    \begin{exampleblock}{\textbf{Desafio 05.}}
        \justifying
        Quebrando aleatoriamente um macarrão, de tamanho qualquer, em três partes, qual a probabilidade de que elas possam formar um triângulo?

        \begin{center}
            \includegraphics[scale=0.3]{imagens/macarrao_rmvbg.png}
        \end{center}
    
    \end{exampleblock}
   
\end{frame}

\section{Referências}

\begin{frame}{Referências}
     \nocite{*}
    \printbibliography
\end{frame}

\end{document}