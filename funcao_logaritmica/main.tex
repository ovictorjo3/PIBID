\documentclass[12pt]{article}
\usepackage[brazil]{babel}
\usepackage[utf8]{inputenc}
\usepackage{amsmath}
\usepackage{amsfonts}
\usepackage{amssymb}
\usepackage{geometry}
\usepackage{graphicx}
\usepackage{float}
\usepackage{enumitem}
\usepackage{multicol}
\usepackage{array}
\usepackage{xcolor}
\usepackage{siunitx}
\usepackage{url}


\usepackage[T1]{fontenc}
\usepackage{mathptmx} %times new roman


\geometry{a4paper, margin=1.2cm}
\setlength{\parindent}{0cm}

%\newcommand{\vermelho}[1]{{\color{red}#1}}
% implementa contador
% Definindo o contador e o comando de questão
\newcounter{questao}
\newcommand{\novaquestao}[1]{%
  \stepcounter{questao}%
  \subsection*{Questão \thequestao\ (#1)}%
}

\title{LISTA DE EXERCÍCIOS SOBRE EQUAÇÃO E FUNÇÃO LOGARÍTMICA}
\author{CETI BILÍNGUE GILBERTO MESTRINHO DE MEDEIROS RAPOSO}
\date{}

\begin{document}
    % Cabeçalho reduzido
    \thispagestyle{empty}
    \vspace{0.5cm}
    \begin{center}
        \large
        \begin{tabular}{|l l|}
            \hline
            \textbf{ESCOLA:} & EETI GILBERTO MESTRINHO DE MEDEIROS RAPOSO \\ 
            \textbf{ALUNA(O):} & \underline{\hspace{7cm}} \textbf{SÉRIE:} \underline{\hspace{1.5cm}} \textbf{TURMA:} \underline{\hspace{1.5cm}} \\
            \textbf{PROFESSOR:} & \underline{\hspace{7cm}} \textbf{DATA:} \underline{\hspace{1.5cm}}/\underline{\hspace{1.5cm}}/\underline{\hspace{1.5cm}} \\
            \textbf{VALOR:} & \underline{\hspace{3cm}} \textbf{NOTA:} \underline{\hspace{1.5cm}} \\
            \hline
        \end{tabular}
    \end{center}
    \vspace{0.5cm}
    
    % Título manual
    \begin{center}
        \Large\textbf{LISTA DE EXERCÍCIOS SOBRE EQUAÇÃO E FUNÇÃO LOGARÍTMICA}
    \end{center}
    
    \vspace{0.3cm}
    
    \section*{ATENÇÃO:}
    \begin{itemize}[noitemsep]
        \item Resolva toda a lista, justificando cada questão.
        \item Colocar o nome completo e identificação no cabeçalho.
        \item Faça na lista, se e somente se a resolução de cada questão couber em cada questão.
        \item Há apenas uma opção correta em cada questão de múltipla escolha.
        \item Caso opte por fazer numa folha à parte, identifique cada questão.
    \end{itemize}
    % Início das colunas com linha vertical
    %\mostravermelhotrue
    \begin{multicols}{2}
        \columnseprule=0.4pt
        \columnsep=20pt
        \large
        
        \novaquestao{UFAM - PSC 2025} O nível de intensidade sonora é medido em decibéis(dB) e segue a fórmula:
        
        	\begin{center}
        		$L=10 \ . \ \log\left(\dfrac{I}{I_{0}}\right)$    
        	\end{center} onde:
        
        	\begin{itemize}
        		\item L é o nível sonoro em decibéis.
        		\item I é a intensidade do som.
        		\item $I_{0}$ é a intensidade mínima perceptível pelo ouvido humano.
        	\end{itemize} Em um dia tranquilo, o ruído em uma biblioteca foi medido em \textbf{40 dB}, enquanto, em uma avenida movimentada, o ruído atingiu \textbf{70 dB}. Com base nessa informação, é correto afirmar que o som na avenida movimentada é
        
        	\begin{enumerate}[label=(\Alph*), noitemsep]
        		\item $1000$ vezes maior que o da biblioteca. %
        		\item $1,75$ vezes maior que o da biblioteca.
        		\item $3$ vezes maior que o da biblioteca.
        		\item $10$ vezes maior que o da biblioteca.
        		\item $100$ vezes maior que o da biblioteca.
        	\end{enumerate}
        
        \novaquestao{UFAM - PSC 2024} Estudos demográficos revelam que a população de certo país, no ano zero, é $f_{0}$ e, decorridos $t$ anos, a população poderá ser estimada pela função:
        
        	\begin{center}
        		$f\left( t\right)=f_{0}\ .\ e^{0,05t}$
        	\end{center} Considerando $\ln(3) = 1,10$, podemos afirmar que a população desse país deverá triplicar quando decorrerem, aproximadamente,
        
        	\begin{enumerate}[label=(\Alph*), noitemsep]
        		\item 10 anos
        		\item 16 anos
        		\item 18 anos
        		\item 20 anos
        		\item 22 anos %
        	\end{enumerate}
        	
        \novaquestao{UFAM - PSC 2007} Considere as funções $f\left( x\right)=\log_3\left( 9x^{2}\right)$ e $g\left( x\right)=\log_3\left({1}/{x}\right)$, definidas para todo $x > 0$. Então, $1+f(x)+g(x)$ é igual a:
        
        	\begin{enumerate}[label=(\Alph*), noitemsep]
        		\item $1+\log_3(x)$
        		\item $3+\log_3(x)$ %
        		\item $3-\log_3(x)$
        		\item $1-\log_3(x)$
        		\item $3\log_3(x)$
        	\end{enumerate}
        
        \novaquestao{UFAM - PSC 2008} Dada a equação $\log_3(x)+\log_3(x^{2})+\log_3(x^{3})+...+\log_3(x^{40})=2460$. Então x é igual a:
        
        	\begin{enumerate}[label=(\Alph*), noitemsep]
        		\item $81$
        		\item $27$ %
        		\item $9$
        		\item $3$ 
        		\item $243$
        	\end{enumerate}
        
        \novaquestao{UFAM 2023} O domínio da função $f(x)=\log_{(x-5)}(x+3)$ é o conjunto:
        
        	\begin{enumerate}[label=(\Alph*), noitemsep]
        		\item $D(f)=\left\{x \in \mathbb{R} \ | \ x \ge 3 \ \text{e} \ x \neq -3 \right\}$
        		\item $D(f)=\left\{x \in \mathbb{R} \ | \ x>5 \ \text{e} \ x \neq 6 \right\}$
        		\item $D(f)=\left\{x \in \mathbb{R} \ | \ x \ge 5 \ \text{e} \ x \neq 6 \right\}$
        		\item $D(f)=\left\{x \in \mathbb{R} \ | \ x>5 \ \text{e} \ x=-3 \right\}$
        		\item $D(f)=\left\{x \in \mathbb{R} \ | \ x>3 \ \text{e} \ x \neq 6 \right\}$
        	\end{enumerate}
        
        \novaquestao{UEA} Uma planta, inicialmente com 15 cm de altura, teve seu crescimento observado durante 90 dias. Nesse período seu crescimento obedeceu à função $f\left( x\right)=15+\log_2(x)$, sendo $f\left( x\right)$ a altura, em centímetros, e x o número de dias, com $1 \le x \le 90$. O número de dias necessários para que a altura dessa planta chegasse a 21 cm foi:
        
        	\begin{enumerate}[label=(\Alph*), noitemsep]
        		\item 62
        		\item 66
        		\item 60
        		\item 68
        		\item 64 %
        	\end{enumerate}
        
        \novaquestao{UEA} Ao estudar um exemplar de uma espécie de peixe ornamental, os pesquisadores constataram que, no 1º dia de observação, o comprimento do peixe era de 2 cm e que, até o 10º dia de observação, o comprimento desse peixe obedeceu à função $y=2+\log_2(x)$, sendo y o comprimento, em cm, e x o número de dias, com $1 \le x \le 10$.Usando $\log(2) \approx 0,30$ e $\log(3) \approx 0,48$, é correto afirmar que o comprimento do peixe, em cm, no 6º dia, era:
        
        	\begin{enumerate}[label=(\Alph*), noitemsep]
        		\item 4,8
        		\item 4,6 %
        		\item 4,4
        		\item 4,2
        		\item 4,0
        	\end{enumerate}
        	
        	
        \novaquestao{UEA 2021} Determinado tipo de colônia de bactérias foi observada por 12 dias. Nesse período, o crescimento da colônia obedeceu à função $f(x)=\text{a}+2\log_{10}(x)$, sendo a um número real não nulo, $f(x)$ o número de bactérias da colônia, em milhares, e x o número de dias, com $x \in \left[1,\ 12\right]$. Se no 1º dia de observação havia 1000 bactérias, o número de bactérias será de 3000 no dia:
        
        
        	\begin{enumerate}[label=(\Alph*), noitemsep]
        		\item 9
        		\item 8
        		\item 10 %
        		\item 7
        		\item 6
        	\end{enumerate}
        	
        \novaquestao{UEA-SIS 2022} Em um sistema de coordenadas cartesianas, considere o gráfico da função $f(x)=\log_{2}(x)$, um ponto A de abcissa ${1}/{2}$ e os pontos B e C, ambos de abcissa 8, conforme mostra a figura.
        
        	\begin{center}
        		\includegraphics[scale=0.4]{imagens/q-uea-sis2022.png}
        	\end{center} Nesse sistema, sabendo que os pontos A e C têm a mesma ordenada, a diferença entre as ordenadas dos pontos B e C é:
        	
        	\begin{enumerate}[label=(\Alph*), noitemsep]
        		\item 2,5
        		\item 3
        		\item 3,5
        		\item 4 %
        		\item 4,5
        	\end{enumerate}
        	
        \novaquestao{ENEM 2016} Uma liga metálica sai do forno a uma temperatura de $ \qty{3.000}{\degreeCelsius}$ e diminui 1\% de sua temperatura a cada 30 min. Use $0,477$ como aproximação para $\log_{10}(3)$ e $1,041$ como aproximação para $\log_{10}(11)$. O tempo decorrido, em hora, até que a liga atinja $\qty{30}{\degreeCelsius}$ é mais próximo de:
        
        \begin{enumerate}[label=(\Alph*), noitemsep]
        	\item 22
        	\item 50
        	\item 100
        	\item 200 %
        	\item 400
        \end{enumerate}
        
        \novaquestao{ENEM 2017} Para realizar a viagem dos
        sonhos, uma pessoa precisava fazer um empréstimo no
        valor de R\$ 5.000, 00. Para pagar as prestações, dispõe de, no
        máximo, R\$ 400,00 mensais. Para esse valor de empréstimo,
        o valor da prestação (P) é calculado em função do número
        de prestações (n) segundo a fórmula:
        
        \begin{center}
        	$P=\dfrac{5.000 \times 1,013^{n} \times 0,013}{\left( 1,013^{n}-1\right)}$
        \end{center} Se necessário, utilize 0,005 como aproximação para $\log(1,013)$; 2,602 como aproximação para $\log(400)$; 2,525 como
        aproximação para $\log(335).$ \\ \
        
        De acordo com a fórmula dada, o menor número de parcelas cujos valores não comprometem o limite definido pela pessoa é:
        
        \begin{enumerate}[label=(\Alph*), noitemsep]
        	\item 12
        	\item 14
        	\item 15
        	\item 16 %
        	\item 17 
        \end{enumerate}
        
        \novaquestao{ENEM 2016} Em 2011, um terremoto de magnitude 9,0 na escala Richter causou um devastador tsunami no Japão, provocando um alerta na usina nuclear de
        Fukushima. Em 2013, outro terremoto, de magnitude 7,0 na mesma escala, sacudiu Sichuan (sudoeste da China), deixando centenas de mortos e milhares de feridos. A magnitude de um terremoto na escala Richter pode ser calculada por:
        
        \begin{center}
        	$M=\dfrac{2}{3}\log\left(\dfrac{E}{E_{0}}\right)$,
        \end{center}
        
        \begin{flushleft}
        	\noindent
        	Disponível em: \url{www.terra.com.br}. Acesso em: 15 ago. 2013 (adaptado)
        \end{flushleft} Qual a relação entre $E_{1}$ e $E_{2}$?
        
        \begin{enumerate}[label=(\Alph*), noitemsep]
        	\item $E_{1}=E_{2}+2$
        	\item $E_{1}=10^{2}E_{2}$
        	\item $E_{1}=10^{3}E_{2}$ %
        	\item $E_{1}=10^{\frac{9}{7}}E_{2}$
        	\item $E_{1}={9}/{7}E_{2}$
        \end{enumerate}
        
        \novaquestao{UFRN} Na figura abaixo, estão esboçados os gráficos das funções $y = \log_3(x)$ e $y=x$. O gráfico da função que está representado em azul é simétrico ao gráfico da função $y=\log_3(x)$ em relação à reta de equação $y=x$. A função que corresponde ao gráfico azul é:
        
        \begin{center}
        	\includegraphics[scale=0.7]{imagens/q-ufrn.png}
        \end{center}
        
        \begin{enumerate}[label=(\Alph*), noitemsep]
        	\item $y={x}/{3}$
        	\item $y=3x$
        	\item $y=x^{3}$
        	\item $y=3^{x}$ %
        	\item $y=3x+3$
        \end{enumerate}
        
        
        \novaquestao{ENEM 2015} Um engenheiro projetou um automóvel cujos vidros das portas dianteiras foram desenhados de forma que suas bordas superiores fossem representadas pela curva de equação $y=\log(x)$, conforme a figura.
        
        	\begin{center}
        		\includegraphics[scale=0.6]{imagens/q-enem_janela.png}
        	\end{center} A forma do vidro foi concebida de modo que o eixo x sempre divida ao meio a altura h do vidro e a base do vidro seja paralela ao eixo x. Obedecendo a essas condições, o
        	engenheiro determinou uma expressão que fornece a altura h do vidro em função da medida n de sua base, em metros.  \\ \ 
        	
        	
        	A expressão algébrica que determina a altura do vidro é:
        	
        	\begin{enumerate}[label=(\Alph*), noitemsep]
        		\item $2\log\left(\dfrac{n+\sqrt{n^{2}+4}}{2}\right) $ \\   %
        		\item $\log\left( \dfrac{n+\sqrt{n^{2}+4}}{2}\right) -  \log\left( \dfrac{n+\sqrt{n^{2}-4}}{2}\right)$ \\
        		\item $\log\left( \dfrac{n+\sqrt{n^{2}+4}}{2}\right)$ \\
        		\item $\log\left(1+\dfrac{n}{2}\right) + \log\left(1-\dfrac{n}{2}\right) $ \\
         		\item $\log\left(1+\dfrac{n}{2}\right) - \log\left(1-\dfrac{n}{2}\right) $
        	\end{enumerate}
        	

          
    
          \novaquestao{UFPR 2019} Um tanque contém uma solução de água e sal cuja concentração está diminuindo devido à adição de mais água. Suponha que a concentração Q(t) de sal no tanque, em gramas por litro (g/L), decorridas
          t horas após o início da diluição, seja dada por
          
          	\begin{center}
          		$Q(t)=100\ . \ 5^{-0,3t}$
          	\end{center} Assinale a alternativa que mais se aproxima do tempo
          	necessário para que a concentração de sal diminua para 50 g/L. Dado: use $\log5 = 0,7$
          	
          	\begin{enumerate}[label=(\Alph*), noitemsep]
          		\item 4 horas e 45 minutos
          		\item 3 horas e 20 minutos
          		\item 2 horas e 20 minutos
          		\item 1 horas e 25 minutos 
          		\item 20 minutos
          	\end{enumerate}
          	
          	
          \novaquestao{Farias Brito EAD Medicina}  Na figura a seguir, dois vértices do trapézio sombreado estão no eixo \textbf{x} e os outros dois vértices estão sobre o gráfico da função real $f(x) = \log_k(x)$, com $k>0$ e $k \neq 1$. Sabe-se que o trapézio sombreado tem 30 unidades de área; assim, o valor de $k+P-Q$ é:
         
         
         	\begin{center}
         		\includegraphics[scale=0.45]{imagens/q-FB.png}
         	\end{center}
         	
         	\begin{enumerate}[label=(\Alph*), noitemsep]
         		\item $-20$
         		\item $-15$ %
         		\item $10$
         		\item $15$ 
         		\item $20$
         	\end{enumerate}
         
         \novaquestao{Farias Brito EAD Medicina} Em 2017 iniciou-se a ocupação de uma região no interior do país, dando origem a uma pequena cidade. Estima-se que a população dessa cidade tenha crescido segundo a função $P=0,1+\log_2(x-2016)$, onde P é a população no ano \textbf{x}, em milhares de habitantes. Considere $\sqrt{2} \approx 1,41$, podemos concluir que a população dessa cidade atingirá a marca dos 3600 habitantes em meados do ano:
         
         	
         	\begin{enumerate}[label=(\Alph*), noitemsep]
         		\item $2023$
         		\item $2025$ 
         		\item $2027$ %
         		\item $2029$ 
         		\item $2031$
         	\end{enumerate}
         
         \novaquestao{Farias Brito EAD Medicina} Fulano aplicou um capital de R$\$ 15000,00$ a juros compostos, pelo período de 4 anos e a uma taxa de $2\%$ am. Ao contabilizar o valor recebido ao final da aplicação, fulano concluiu que o valor corresponde a: \textbf{Dado}: $\log_{1,02}(1,60)=24$
         
         	\begin{enumerate}[label=(\Alph*), noitemsep]
         		\item R$\$ 36.800,00$
         		\item R$\$ 37.200,00$
         		\item R$\$ 39.700,00$
         		\item R$\$ 37.600,00$
         		\item R$\$ 38.400,00$ %
         	\end{enumerate}
         
         \novaquestao{EsPCEx 2008} O valor de x para o qual as funções reais $f(x)=2^{x}$ e $g(x)=5^{1-x}$ possuem a mesma imagem é:
         
         
         	\begin{enumerate}[label=(\Alph*), noitemsep]
         		\item $\log(2)+1$
         		\item $\log(2)-1$
         		\item $1-\log(2)$ %
         		\item $2\log(2)+1$
         		\item $1-2\log(2)$ 
         	\end{enumerate}
         	
         \novaquestao{UECE} A função $f(x)=\log_2(x)$, denominada de função logaritmo na base 2, é definida para todo número real positivo x. São bem conhecidas, dentre outras, as seguintes propriedades da função $f$: para cada número real positivo $a$ e para todo número inteiro $n$, verificam-se as igualdades $2^{f\left( a\right)}=a$ e $f\left( a^{n}\right)=n.f\left( a\right)$. Com base nestes fatos e em outros conhecimentos básicos de matemática, é possível concluir-se corretamente que $f\left( 0,03125\right)$ é igual a:
         
         	\begin{enumerate}[label=(\Alph*), noitemsep]
         		\item $-5$ %
         		\item $-2$ 
         		\item $2$ 
         		\item $5$ 
         		\item $3$
         	\end{enumerate}
         	
         	
         	
         \novaquestao{UFPA} As populações A e B de duas cidades são determinadas em milhares de habitantes pelas funções: $A(t)=\log_4(2+t)^{5}$ e $B(t)=\log_2(2t+4)^{2}$, nas quais a variável $t$ representa o tempo em anos. Essas cidades terão o mesmo número de habitantes no ano $t$, que é igual a:
         
         	\begin{enumerate}[label=(\Alph*), noitemsep]
         		\item $6$ 
         		\item $8$ 
         		\item $10$ 
         		\item $12$ 
         		\item $14$ %
         	\end{enumerate}
         	
         	
         \novaquestao{PUC-RS} O modelo da cobertura que foi colocada no Estádio Beira-Rio está representado na figura abaixo.
         
         	\begin{center}
         		\includegraphics[scale=0.7]{imagens/q-pucrs.png}
         	\end{center} Colocada devidamente em um plano cartesiano, é possível afirmar que, na forma em que está, a linha em destaque pode ser considerada uma restrição da representação da função dada por:
         
         	\begin{enumerate}[label=(\Alph*), noitemsep]
         		\item $y=\log(x)$ % 
         		\item $y=x^{2}$ 
         		\item $y=|x|$ 
         		\item $y=\sqrt{-x}$ 
         		\item $y=10^{x}$ 
         	\end{enumerate}
         	
         	
         \novaquestao{Unisc-RS 2022} Determinada espécie de eucalipto apresenta uma relação que interliga seu tamanho (altura) com seu tempo de plantio, dada por $h(t)=26+\log_3(1,5t)$, em que h(t) é a altura dada em metros, e t indica o tem-
         po em anos. Nesse caso, qual é o tempo necessário (em anos) para que a árvore de eucalipto atinja a altura de 28 m?
         
         	\begin{enumerate}[label=(\Alph*), noitemsep]
         		\item 4
         		\item 7
         		\item 2
         		\item 5
         		\item 6 %
         	\end{enumerate}
         	
         \novaquestao{UFRGS 2023} O valor de $\log2^{2}+\log2^{3}+\log2^{4}+...+\log2^{50}$ é:
         
         
         	\begin{enumerate}[label=(\Alph*), noitemsep]
         		\item $\log2^{1247}$
         		\item $\log2^{1274}$ %
         		\item $\log2^{1472}$
         		\item $\log2^{59}$
         		\item $\log8^{59}$
         	\end{enumerate}
         	
         \novaquestao{EsSA-MG 2023} A altitude (h) acima do nível do mar, em quilômetros, durante o voo de um avião é dada em função da pressão atmosférica p, em atm, por $h(p)=30.\log_{10}\left( \dfrac{1}{p} \right) $. Em determinado instante, a pressão atmosférica medida pelo altímetro desse avião era de 0,8 atm. Nesse instante, a altitude do avião, em quilômetros, considerando $\log_{10}(2)=0,3$, era de:
         
         	\begin{enumerate}[label=(\Alph*), noitemsep]
         		\item 2
         		\item 3
         		\item 6
         		\item 8
         		\item 9
         	\end{enumerate}
         	
         	
         
         
         
         
         
         	
         	
         	
         	
         	
         	
    \end{multicols}
    
\end{document}