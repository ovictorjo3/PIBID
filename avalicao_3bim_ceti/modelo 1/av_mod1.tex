
\documentclass[12pt]{article}
\usepackage[brazil]{babel}
\usepackage[utf8]{inputenc}
\usepackage{amsmath}
\usepackage{amsfonts}
\usepackage{amssymb}
\usepackage{geometry}
\usepackage{graphicx}
\usepackage{float}
\usepackage{enumitem}
\usepackage{multicol}
\usepackage{array}
\usepackage{xcolor}
\usepackage[T1]{fontenc}
\usepackage{mathptmx} %times new roman


\geometry{a4paper, margin=1.2cm}
\setlength{\parindent}{0cm}

% implementa contador
% Definindo o contador e o comando de questão
\newcounter{questao}
\newcommand{\novaquestao}[1]{%
	\stepcounter{questao}%
	\subsection*{Questão \thequestao\ (#1)}%
}

\begin{document}
	\large
	\thispagestyle{empty}
	% Cabeçalho reduzido
	\vspace{0.5cm}
	\begin{center}
		\large
		\begin{tabular}{|l l|}
			\hline
			\textbf{ESCOLA:} & EETI GILBERTO MESTRINHO DE MEDEIROS RAPOSO \\ 
			\textbf{ALUNA(O):} & \underline{\hspace{7cm}} \textbf{SÉRIE:} \underline{\hspace{1.5cm}} \textbf{TURMA:} \underline{\hspace{1.5cm}} \\
			\textbf{PROFESSOR:} & \underline{\hspace{7cm}} \textbf{DATA:} \underline{\hspace{1.5cm}}/\underline{\hspace{1.5cm}}/\underline{\hspace{1.5cm}} \\
			\textbf{VALOR:} & \underline{\hspace{3cm}} \textbf{NOTA:} \underline{\hspace{1.5cm}} \\
			\hline
		\end{tabular}
	\end{center}
	\vspace{0.5cm}
	
	% Título manual
	\begin{center}
		\Large\textbf{AV1 - 3º BIMESTRE}
	\end{center}
	
	\vspace{0.3cm}
	
	\section*{\textbf{ATENÇÃO:}}
	\begin{itemize}[noitemsep]
		\item Resolva toda a PROVA, justificando cada questão.
		\item Colocar o nome completo e identificação no cabeçalho.
		\item Há apenas uma opção correta em cada questão de múltipla escolha.
	\end{itemize}
	% Início das colunas com linha vertical
	%\mostravermelhotrue
	\begin{multicols}{2}
		\columnseprule=0.4pt
		\columnsep=20pt
		
		\novaquestao{Enem 2023}
		
			No alojamento de uma universidade, há alguns quartos com o padrão superior ao dos demais. Um desses quartos ficou disponível, e muitos estudantes se candidataram para morar no local. Para escolher quem ficará com o quarto, um sorteio será realizado. Para esse sorteio, cartões individuais com os nomes de todos os estudantes inscritos serão depositados em uma urna, sendo que, para cada estudante de primeiro ano, será depositado um único cartão com seu nome; para cada estudante de segundo ano, dois cartões com seu nome; e, para cada estudante de terceiro ano, três cartões com seu nome. Foram inscritos 200 estudantes de primeiro ano, 150 de  segundo ano e 100 de terceiro ano. Todos os cartões têm a mesma probabilidade de serem sorteados. Qual a probabilidade de o vencedor do sorteio ser um estudante de terceiro ano?
		
			\begin{enumerate}[label=(\alph*), noitemsep]
				\item {1}/{2}
				\item {1}/{3}
				\item {1}/{8}
				\item {2}/{9}
				\item {3}/{8}
			\end{enumerate}
		
		\novaquestao{Enem 2019}
		
			Em um determinado ano, os computadores da receita federal de um país identificaram como inconsistentes 20\% das declarações de imposto de renda que lhe foram encaminhadas. Uma declaração é classificada como inconsistente quando apresenta algum tipo de erro ou conflito nas informações prestadas. Essas declarações 
			consideradas inconsistentes foram analisadas pelos auditores, que constataram que 25\% delas eram fraudulentas. Constatou-se ainda que, dentre as declarações que não apresentaram inconsistências, 6,25\% eram fraudulentas. Qual é a probabilidade de, nesse ano, a declaração de um contribuinte ser considerada inconsistente, dado 
			que ela era fraudulenta?
		
			\begin{enumerate}[label=(\alph*), noitemsep]
				\item 0,0500
				\item 0,1000
				\item 0,1125
				\item 0,3125
				\item 0,5000
			\end{enumerate}
		
		\novaquestao{UEA SIS II 2016}
		
			Em uma urna há 20 bolas numeradas de 20 a 39. Retirando-se aleatoriamente uma bola dessa urna, a probabilidade de que o número da bola seja múltiplo de 3 e que a soma dos algarismos seja menor ou igual a 7 é:
			
			\begin{enumerate}[label=(\alph*), noitemsep]
				\item {3}/{5}
				\item {2}/{5} 
				\item {1}/{5} %
				\item {3}/{20} 
				\item {1}/{20}
			\end{enumerate}
		
		\novaquestao{UEA SIS II 2012}
		
			Em um cesto há 250 camu-camus, dos quais 20\% estão verdes
			e 500 acerolas, das quais 15\% também estão verdes. Se uma pessoa
			retirar ao acaso um fruto desses cesto, a probabilidade de que o 
			fruto esteja verde é:
			
			\begin{enumerate}[label=(\alph*), noitemsep]
				\item {2}/{3}
				\item {1}/{3}
				\item {1}/{4}
				\item {1}/{5}  
				\item {1}/{6} %
			\end{enumerate}
		
		\novaquestao{UFU-MG 2018}
		
			As irmãs Ana e Beatriz e seus respectivos namorados vão sentar-se em um banco de 
			jardim (figura) de modo que cada namorado fique ao lado de sua namorada.
			
			\begin{center}
				\includegraphics[scale=0.6]{imagens/ufu-2018.png}
			\end{center} A probabilidade de as irmãs sentarem-se uma ao lado da outra é igual a:
			
			\begin{enumerate}[label=(\alph*), noitemsep]
				\item 0,25
				\item 0,33
				\item 0,45
				\item 0,50
				\item 0,75
			\end{enumerate}
		
		
		
		
		
	\end{multicols}
	
\end{document}