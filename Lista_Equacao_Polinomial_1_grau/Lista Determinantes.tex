\documentclass{article}
\usepackage[utf8]{inputenc}
\usepackage{amsmath}
\usepackage{amsfonts}
\usepackage{amssymb}
\usepackage{geometry}
\usepackage{graphicx}
\usepackage{float}
\usepackage{enumitem}
\usepackage{multicol}

\geometry{a4paper, margin=1cm}

\title{ATIVIDADE AVALIATIVA DE MATEMÁTICA}
\author{CETI BILÍNGUE GILBERTO MESTRINHO DE MEDEIROS RAPOSO}
\date{}

\begin{document}
	\Large
	% Cabeçalho
	\vspace{1cm}
	\begin{center}
		\begin{tabular}{|l l|}
			\hline
			\textbf{ESCOLA:} & EETI GILBERTO MESTRINHO DE MEDEIROS RAPOSO \\
			\textbf{ALUNA(O):} & \underline{\hspace{8cm}} \textbf{SÉRIE:} \underline{\hspace{2cm}} \textbf{TURMA:} \underline{\hspace{2cm}} \\
			\textbf{PROFESSOR:} & \underline{\hspace{8cm}} \textbf{DATA:} \underline{\hspace{1.66cm}}/\underline{\hspace{2cm}}/\underline{\hspace{2cm}} \\
			\textbf{VALOR:} & \underline{\hspace{4cm}} \textbf{NOTA:} \underline{\hspace{2cm}} \\
			\hline
		\end{tabular}
	\end{center}
	\vspace{1cm}
	
	% Título manual
	\begin{center}
		\Large\textbf{LISTA DE EXERCÍCIOS SOBRE DETERMINANTES}
	\end{center}
	
	\vspace{0.5cm}
	
	\section*{ATENÇÃO:}
	\begin{itemize}[noitemsep]
		\item Resolva toda a lista, justificando cada questão.
		\item Colocar o nome completo e identificação no cabeçalho.
		\item Faça na lista, se e somente se a resolução de cada questão couber em cada questão.
		\item Há apenas uma opção correta em cada questão de múltipla escolha.
		\item Caso opte por fazer numa folha à parte, identifique cada questão.
	\end{itemize}
	
	% Início das colunas com linha vertical
	\begin{multicols}{2}
		\columnseprule=0.4pt
		\columnsep=20pt
		
	\section*{Questões}
		
		% Questões 1 a 21
        
        \section*{Questão 1 (PSC – UFAM)}
		Um capital de R$\$ 2.400,00$ foi aplicado a juros simples com taxa de 5$\%$ ao mês. Qual será o montante após 4 meses? 
		
		\begin{enumerate}[label=(\Alph*), noitemsep]
			\item R$\$$ 2.880,00 %
			\item R$\$$ 3.000,00
			\item R$\$$ 3.120,00
			\item R$\$$ 3.120,00
			\item R$\$$ 3.600,00
		\end{enumerate}
        
		\subsection*{Questão 2 (SIS - UEA)}
		
		Uma pessoa emprestou R$\$ 1.500,00$ a uma taxa de juros simples de 3$\%$ ao mês. Em quantos meses o montante será R$\$ 1.950,00$?
		\begin{enumerate}[label=(\Alph*), noitemsep]
			\item 6 meses
			\item 8 meses 
			\item 10 meses %
			\item 12 meses
			\item 15 meses
		\end{enumerate}
		
	
        \subsection*{Questão 3 (SIS - UEA)}
         O salário bruto mensal de Juca é composto de duas partes: uma parte fixa e outra variável, correspondente a uma comissão de 6$\%$ sobre o valor das vendas efetuadas no período. Em certo mês, Juca achou que receberia y reais, mas recebeu apenas 90$\%$ desse valor. Constatou-se, então, que a comissão referente a uma venda no valor de R$\ 7.000,00$ havia sido computada equivocadamente para outro vendedor. Nessas condições, pode-se afirmar que o valor correto do salário bruto de Juca nesse mês era
		
		\begin{enumerate}[label=(\Alph*), noitemsep]
			\item 3.820,00.
			\item 4.200,00.
			\item 3.750,00.
			\item 3.600,00.
			\item 4.420,00.
		\end{enumerate}
		
		\subsection*{Questão 3 (PSC – UFAM 2018)}
		Considere a matriz quadrada \( A = (a_{ij}) \), de ordem 3, onde
		\[
		a_{ij} = \begin{cases} 
			i - 2j, & \text{se } i > j \\
			2i - j, & \text{se } i = j \\
			j - i, & \text{se } i < j 
		\end{cases}
		\]
		O valor do determinante de \( A \) é:
		
		\begin{enumerate}[label=(\Alph*), noitemsep]
			\item 2
			\item 3
			\item 4
			\item 5
			\item 6
		\end{enumerate}
		
		\subsection*{Questão 4 (PSC – UFAM 2017)}
		Sejam \( A = (a_{ij}) \) e \( B = (b_{ij}) \) duas matrizes quadradas de ordem 2, com \( a_{ij} = 2i - j \) e \( b_{ij} = 2i - 3j \). Se \( C = A + B \), então \( C^{-1} \) é igual a:
		
		\begin{enumerate}[label=(\Alph*), noitemsep]
			\item \(\begin{pmatrix} 1 & 2 \\ 3 & 2 \end{pmatrix}\)
			\item \(\begin{pmatrix} 0 & 1/2 \\ -1/2 & 0 \end{pmatrix}\)
			\item \(\begin{pmatrix} 1 & 2 \\ -2 & 0 \end{pmatrix}\)
			\item \(\begin{pmatrix} 1 & 1 \\ 0 & 1 \end{pmatrix}\)
			\item \(\begin{pmatrix} 0 & 1/4 \\ -1/4 & 0 \end{pmatrix}\)
		\end{enumerate}
		
		\subsection*{Questão 5 (PSC – UFAM 2016)}
		O valor do determinante da matriz  
		\[
		A = \begin{pmatrix} 1 & 0 & 0 & 0 & 0 \\ 0 & -2 & 0 & 0 & 0 \\ 0 & 0 & 1 & -1 & 1 \\ 0 & 0 & 1 & 1 & -1 \\ 0 & 0 & -1 & 1 & 1 \end{pmatrix}
		\]
		é igual a:
		
		\begin{enumerate}[label=(\Alph*), noitemsep]
			\item 8
			\item -8
			\item 2
			\item -2
			\item 0
		\end{enumerate}
		
		\subsection*{Questão 6 (PSC – UFAM 2013)}
		Se \( A = \left( a_{ij} \right)_{3 \times 3} \) é uma matriz real definida por  
		\[
		a_{ij} = \begin{cases} 
			i + j, & \text{se } i > j \\
			2j - i, & \text{se } i = j, \\
			i - j, & \text{se } i < j
		\end{cases}
		\]
		então o determinante da matriz inversa da matriz \( A \) é:  
		
		\begin{enumerate}[label=(\Alph*), noitemsep]
			\item 10
			\item \(\frac{1}{10}\)
			\item \(-\frac{1}{10}\)
			\item \(-\frac{1}{5}\)
			\item \(\frac{1}{\epsilon}\)
		\end{enumerate}
		
		\subsection*{Questão 7 (PSC – UFAM 2010)}
		Seja \( A \) uma matriz quadrada de ordem \( n \), tal que  
		\[
		\det A = k, \text{ com } k \neq 0. 
		\]  
		Sendo \( A^{-1} \), a matriz inversa de \( A \), o valor do \(\det A^{-1}\) é:  
		
		\begin{enumerate}[label=(\Alph*), noitemsep]
			\item \( 2k \)
			\item \( 3k \)
			\item \( \frac{k}{3} \)
			\item \( \frac{k}{2} \)
			\item \( \frac{1}{k} \)
		\end{enumerate}
		
		\subsection*{Questão 8 (SIS – UEA 2017)}
		Considere a matriz \( A = \begin{pmatrix} 1 & 2 \\ 3 & 4 \end{pmatrix} \). O valor do determinante da matriz \( A^2 - 2A \) é igual a:
		
		\begin{enumerate}[label=(\Alph*), noitemsep]
			\item 0
			\item 2
			\item 4
			\item 6
			\item 8
		\end{enumerate}
		
		\subsection*{Questão 9 (SIS – UEA 2018)}
		Dada a matriz quadrada \( A = (a_{ij}) \), de ordem 2, tal que  
		\( a_{ij} = 5i - j^2 \), o valor do determinante da matriz \( A \) é igual a:
		
		\begin{enumerate}[label=(\Alph*), noitemsep]
			\item 5
			\item 9
			\item 15
			\item 19
			\item 25
		\end{enumerate}
		
		\subsection*{Questão 10 (SIS – UEA 2019)}
		Dadas as matrizes \( P = \begin{pmatrix} -3 & m \\ 4 & 5 \end{pmatrix} \) e \( Q = \begin{pmatrix} -1 & 3 & 0 \\ 2 & 1 & 2 \\ 2 & 1 & 1 \end{pmatrix} \), sabe-se que \(\det(P) + \det(Q) = 0\). O valor da constante real \( m \) é:
		
		\begin{enumerate}[label=(\Alph*), noitemsep]
			\item \(-2\)
			\item \(-1\)
			\item 0
			\item 1
			\item 2
		\end{enumerate}
		
		\subsection*{Questão 11 (SIS – UEA 2016)}
		Considere as matrizes \( A = \begin{pmatrix} x & 3 \\ y & 4 \end{pmatrix} \) e \( B = \begin{pmatrix} 1 & -1 & 2 \\ y & 1 & 1 \\ x & 0 & 5 \end{pmatrix} \), com \( x \) e \( y \) números reais. Sabendo que \(\det A = \det B\) e que \( x + y = 5 \), o valor de \( x^y \) é igual a:
		
		\begin{enumerate}[label=(\Alph*), noitemsep]
			\item 1
			\item 2
			\item 6
			\item 8
			\item 9
		\end{enumerate}
		
		\subsection*{Questão 12 (VUNESP)}
		Considere as matrizes \( A = \begin{pmatrix} k & 0 & k \\ 3 & -2 & k \end{pmatrix} \), sendo \( k \) um número real, com \( k < 2 \), \( B = (b_{ij})_{3 \times 2} \), com \( b_{ij} = (i - j)^2 \), \( C = A \cdot B \). Sabendo que \(\det C = 12\), o valor de \( k^2 \) é:
		
		\begin{enumerate}[label=(\Alph*), noitemsep]
			\item 0
			\item 9
			\item 4
			\item 16
			\item 1
		\end{enumerate}
		
		\subsection*{Questão 13 (VUNESP)}
		Considere a matriz quadrada \( A = (a_{ij})_{2 \times 2} \), de ordem 2, tal que  
		\[
		a_{ij} = \begin{cases} 
			\cos^2 (i\pi) + \sin^2 (j\pi), & \text{se } i = j \\
			5|i - j|, & \text{se } i \neq j
		\end{cases}
		\]
		para todo \( i, j \in \{1, 2\} \). Seja \( B = 3A^{-1} \), o determinante da matriz \( B \) vale:
		
		\begin{enumerate}[label=(\Alph*), noitemsep]
			\item \(-\frac{3}{8}\)
			\item 1
			\item \(-\frac{8}{3}\)
			\item \(\frac{8}{3}\)
			\item \(\frac{3}{8}\)
		\end{enumerate}
		
		\subsection*{Questão 14 (VUNESP)}
		Seja \( Q_n \), em que \( n \) é um número inteiro maior que zero, uma matriz quadrada de ordem 2. Uma matriz desse tipo é definida por  
		\[
		Q_n = \begin{pmatrix} 2n^2 & -n \\ n + 1 & 1 \end{pmatrix}
		\]
		O determinante da matriz \( Q_7 \) é igual a:
		
		\begin{enumerate}[label=(\Alph*), noitemsep]
			\item 154
			\item 172
			\item 178
			\item 160
			\item 166
		\end{enumerate}
		
		\subsection*{Questão 15 (VUNESP)}
		Se a matriz \( A = \begin{pmatrix} 2 & K & K \\ 1 & 2 & 1 \\ 1 & 1 & 2 \end{pmatrix} \), então o \(\det A\) é igual a:
		
		\begin{enumerate}[label=(\Alph*), noitemsep]
			\item 8
			\item 12
			\item 4
			\item 16
			\item 6
		\end{enumerate}
		
		\subsection*{Questão 16 (ESPM-SP)}
		Dadas as matrizes \( A = \begin{pmatrix} x & 2 \\ 1 & 1 \end{pmatrix} \) e \( B = \begin{pmatrix} 1 & x \\ -1 & 2 \end{pmatrix} \), a diferença entre os valores de \( x \), tais que det \( (A \cdot B) = 3x \), pode ser igual a:
		
		\begin{enumerate}[label=(\Alph*), noitemsep]
			\item 3
			\item -2
			\item 5
			\item -4
			\item 1
		\end{enumerate}
		
		\subsection*{Questão 17 (Ifal)}
		Se \( A = \begin{pmatrix} 1 & 2 \\ -1 & 0 \end{pmatrix} \) e \( B = \begin{pmatrix} 1 & 2 \\ -1 & 0 \end{pmatrix} \), o determinante da matriz \( (AB)^{-1} \) é:
		
		\begin{enumerate}[label=(\Alph*), noitemsep]
			\item \( -\frac{1}{10} \)
			\item \( \frac{21}{10} \)
			\item \( \frac{13}{10} \)
			\item \( -\frac{13}{10} \)
			\item nda
		\end{enumerate}
		
		\subsection*{Questão 18 (PUC-PR)}
		Considere as seguintes desigualdades:
		
		I. \(\begin{vmatrix} 2 & 2 \\ -1 & 4 \end{vmatrix} > \begin{vmatrix} 3 & 4 \\ 1 & 5 \end{vmatrix}\)
		
		II. \(\begin{vmatrix} 3 & -6 \\ 5 & -2 \end{vmatrix} < \begin{vmatrix} 4 & 7 \\ -1 & 5 \end{vmatrix}\)
		
		III. \(\begin{vmatrix} 8 & 1 \\ -2 & -6 \end{vmatrix} > \begin{vmatrix} 9 & 2 \\ -1 & -7 \end{vmatrix}\)
		
		É correto afirmar que:
		
		\begin{enumerate}[label=(\Alph*), noitemsep]
			\item são verdadeiras apenas as desigualdades I e II.
			\item são verdadeiras apenas as desigualdades II e III.
			\item são verdadeiras apenas as desigualdades I e III.
			\item as três desigualdades são verdadeiras.
			\item as três desigualdades são falsas.
		\end{enumerate}
		
		\subsection*{Questão 19 (Udesc)}
		Dada a matriz \( A = \begin{pmatrix} 1 & 2 \\ 1 & -1 \end{pmatrix} \). Seja a matriz \( B \) tal que \( A^{-1} BA = D \), onde a matriz \( D = \begin{pmatrix} 2 & 1 \\ -1 & 2 \end{pmatrix} \), então o determinante de \( B \) é igual a:
		
		\begin{enumerate}[label=(\Alph*), noitemsep]
			\item 3
			\item -5
			\item 2
			\item 5
			\item -3
		\end{enumerate}
		
		\subsection*{Questão 20 (Furb-SC)}
		Sendo det \(\begin{vmatrix} 2^2 & 1 \\ \log_2 x & 1 \end{vmatrix} = 0\), então, o valor de \( x \) será igual a:
		
		\begin{enumerate}[label=(\Alph*), noitemsep]
			\item 4
			\item 8
			\item 32
			\item 16
		\end{enumerate}
		
		\subsection*{Questão 21 (UEA - SIS 2017)} 
		Uma matriz quadrada \( A \) é chamada matriz diagonal se  
		\[
		a_{ij} = 0 \text{ para } i \neq j.
		\]
		Seja a matriz \( A = \begin{pmatrix} y & 2x + 3y \\ y + 2 & x \end{pmatrix} \), com  
		\( x \) e \( y \) números reais. Sabendo que \( A \) é uma matriz diagonal, seu determinante vale:
		
		\begin{enumerate}[label=(\Alph*), noitemsep]
			\item \(-6\).
			\item \(-1\).
			\item 0.
			\item 1.
			\item 5.
		\end{enumerate}
		
		
		
	\end{multicols}
	
\end{document}